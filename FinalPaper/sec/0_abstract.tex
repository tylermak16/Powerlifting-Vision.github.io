\begin{abstract}
Powerlifting is a sport consisting of three main barbell lifts, the squat, the bench press, and the deadlift. For each movement, there are three attempts to perform. All three of these movements are judged by three judges, two on the sides, and one in the front. These lifts must be performed to a certain standard in order to be called a ``good lift." This paper examines the barbell squat and seeks to utilize existing computer vision technology to judge squat depth. For a squat to be considered ``depth" in Powerlifting, the hip crease must be below the top of the knee. To do this, recent pose estimation models are used to generate pose estimations from videos of a human squatting, as well as key points corresponding to joints and other body parts. The pose estimation video output and key points are then fed into a trained neural network that predicts if a squat meets the depth standard. We demonstrate that our simple framework is effective in judging squat depth, with [add remarks about data here]. \\
\end{abstract}